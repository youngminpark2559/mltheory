\documentclass{article}
\usepackage{amsmath}
\usepackage{kotex}
\usepackage{amssymb}
\usepackage{bigints}
\begin{document}

Probabilistic sample(or random sample, sample): \\ 
One realizable phenomenon from the probabilistic problem you want to solve \\ 
Or one sampled case \\ 

Sample space $\Omega$: \\ 
Set which contains all possible samples \\ 

Task of defining sample sapce: \\
Define which phenomenon is possible to occur \\
and which phenomenon is impossible to occur \\ 

================================================== \\
Sample space when you toss the coin \\ 
$\Omega =\{H,T\}$ \\ 

================================================== \\
Some cases has set of entire real numbers as sample space \\ 
$\Omega = \mathbf{R}$ \\ 

================================================== \\
Possible events: possible sub sets of sample space $\Omega$ \\ 

Sample space $\Omega=\{H,T\}$ \\ 
Possible events: $\phi,\{H\},\{T\},\{H,T\}$

================================================== \\
Probability is function which takes all events \\
and which outputs number \\ 

$P(A)=0.1$ \\
P() is function P \\
A is event \\
0.1 is probability value \\

================================================== \\
Kolmogorov's axioms \\ 

1. $P(\text{all\_events})\ge 0$ \\ 
2. $P(\Omega)=1$ \\ 
3. If $A\cap B=\phi$, then $P(A\cup B)=P(A)+P(B)$ \\ 

================================================== \\
Interpretion about probability value: \\ 
1. Frequentist \\ 
2. Baysian \\ 

================================================== \\

\end{document}
