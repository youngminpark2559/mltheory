\documentclass{article}
\usepackage{amsmath}
\usepackage{kotex}
\usepackage{amssymb}
\usepackage{bigints}
\begin{document}

Set: distinguishable collection. \\ 
Element: distinguishable object in the set \\ 

Element x is in set A: \\ 
$x\in A$ \\ 

================================================== \\
Cardinality is the number of element which the set has \\ 

$A=\{1,2,3\}$ \\ 
$|A| = \text{card}(A) = 3$ \\ 

================================================== \\
Union of two sets A and B \\ 
$A\cup B$ \\ 

Intersection of two sets A and B \\ 
$A\cap B$ \\ 

================================================== \\
If set A is sub set of set B: \\ 
$A \subset B$ \\ 

$A \subset A$ is true \\

If cardinality of A is smaller than B, \\ 
A is proper subset of B \\ 

================================================== \\
Difference set \\ 
$A-B$ \\ 

================================================== \\
Complement set against entire set $\Omega$ \\ 
$A^C$ \\ 

$A^C=\Omega-A$ \\ 

================================================== \\
Null set $\phi$ has none element \\ 

$\phi \subset$ all sets \\ 

Following is true \\ 
$A\cap \phi=\phi$ \\ 
$A\cup \phi=A$ \\ 
$A\cap A^C=\phi$ \\ 

================================================== \\
$A=\{1,2\}$ has 4 number of sub sets \\ 

$A_1=\phi$ \\
$A_2=\{1\}$ \\
$A_3=\{2\}$ \\
$A_4=\{1,2\}$ \\

================================================== \\
If set A has N number of elements, \\ 
A has $2^N$ number of sub sets \\ 

================================================== \\
Following is true as distribution law \\ 
$A\cup(B\cap C)=(A\cup B)\cap(A\cup C)$ \\ 
$A\cap(B\cup C)=(A\cap B)\cup(A\cap C)$ \\ 

================================================== \\

afafaf
Probabilistic sample(or random sample, sample): \\ 
One realizable phenomenon from the probabilistic problem you want to solve \\ 
Or one sampled case \\ 

Sample space $\Omega$: \\ 
Set which contains all possible samples \\ 

Task of defining sample sapce: \\
Define which phenomenon is possible to occur \\
and which phenomenon is impossible to occur \\ 

================================================== \\
Sample space when you toss the coin \\ 
$\Omega =\{H,T\}$ \\ 

================================================== \\
Some cases has set of entire real numbers as sample space \\ 
$\Omega = \mathbf{R}$ \\ 

================================================== \\
Possible events: possible sub sets of sample space $\Omega$ \\ 

Sample space $\Omega=\{H,T\}$ \\ 
Possible events: $\phi,\{H\},\{T\},\{H,T\}$

================================================== \\
Probability is function which takes all events \\
and which outputs number \\ 

$P(A)=0.1$ \\
P() is function P \\
A is event \\
0.1 is probability value \\

================================================== \\
Kolmogorov's axioms \\ 

1. $P(\text{all\_events})\ge 0$ \\ 
2. $P(\Omega)=1$ \\ 
3. If $A\cap B=\phi$, then $P(A\cup B)=P(A)+P(B)$ \\ 

================================================== \\
Interpretion about probability value: \\ 
1. Frequentist \\ 
2. Baysian \\ 

================================================== \\

\end{document}
