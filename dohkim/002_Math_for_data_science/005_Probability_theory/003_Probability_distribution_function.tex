\documentclass{article}
\usepackage{amsmath}
\usepackage{kotex}
\usepackage{amssymb}
\usepackage{bigints}
\begin{document}
================================================== \\
Probabilistic sample(or random sample, sample): \\ 
One realizable phenomenon from the probabilistic problem you want to solve \\ 
Or one sampled case \\ 

Sample space $\Omega$: \\ 
Set which contains all possible samples \\ 

Task of defining sample sapce: \\
Define which phenomenon is possible to occur \\
and which phenomenon is impossible to occur \\ 

================================================== \\
Sample space when you toss the coin \\ 
$\Omega =\{H,T\}$ \\ 

================================================== \\
Some cases has set of entire real numbers as sample space \\ 
$\Omega = \mathbf{R}$ \\ 

================================================== \\
Possible events: possible sub sets of sample space $\Omega$ \\ 

Sample space $\Omega=\{H,T\}$ \\ 
Possible events: $\phi,\{H\},\{T\},\{H,T\}$

================================================== \\
Probability is function which takes all events \\
and which outputs number \\ 

$P(A)=0.1$ \\
P() is function P \\
A is event \\
0.1 is probability value \\

================================================== \\
Kolmogorov's axioms \\ 

1. $P(\text{all\_events})\ge 0$ \\ 
2. $P(\Omega)=1$ \\ 
3. If $A\cap B=\phi$, then $P(A\cup B)=P(A)+P(B)$ \\ 

================================================== \\
Interpretion about probability value: \\ 
1. Frequentist \\ 
2. Baysian \\ 

================================================== \\
afafaf

Information of "event A has high probability value", \\ 
"event B has medium probability value", \\ 
"event C has low probability value", ... \\ 
is called "probability distribution" \\ 

================================================== \\
Elementary event (or atomic event): \\ 
Elementary event has one number of sample \\ 

$P({\text{Diamond}})=0.1$ \\ 
$P({\text{Heart}})=0.2$ \\ 
$P({\text{Spade}})=0.3$ \\ 
$P({\text{Clover}})=0.4$ \\ 

Then, you can calculate probability value of all kinds of events, \\ 
which has 2 samples, 3 samples, etc \\ 
according to 3rd rule of Kolmogorov's axiom \\ 

$P({\text{Heart,Spade}})=0.2+0.3=0.5$ \\ 

================================================== \\
Probability mass function: \\ 
Probability mass function defines probability values \\
to each elementary event, \\ 
when there are only finite number of events \\ 

================================================== \\
$P(\{1\})=0.2$ \\ 
$P()$ is probability function \\ 
$\{1\}$ is event which has one sample \\ 
$0.2$ is probability value for event $\{1\}$ \\ 

$p(1)=0.2$ \\ 
$p()$ is probability mass function \\ 
$1$ is elementary event which has number 1 \\ 
$0.2$ is probability value for elementary event 1 \\ 

================================================== \\
$P(\{1,2\})=0.2$ \\ 
$P()$ is probability function \\ 
$\{1,2\}$ is event which has 2 samples \\ 
$0.2$ is probability value for event $\{1,2\}$ \\ 

$p(1,2)$ can be defined \\ 

================================================== \\
Simple interval event A: $A=\{a\le x <b\}$ \\ 

$A = \{ a \leq x < b \} \;\; \rightarrow \;\; P(A) = P(\{ a \leq x < b \}) = P(a, b)$ \\ 

$P(B) = P(\{ -2 \leq x < 1\}) + P(\{2 \leq x < 3\}) = P(-2, 1) + P(2, 3)$ \\ 

$P(B) = P(\{ -2 \leq x < 3 \}) - P(\{ 1 \leq x < 2\}) = P(-2, 3) - P(1, 2)$ \\ 

================================================== \\
Cumulative distribution function: \\ 
From above, you used 2 numbers to define "interval" or "simple interval event A" \\ 

To use only one number, you can use negative infinity \\ 

$
S_{-1} = \{ -\infty \leq X < -1 \} \\
S_{0} = \{ -\infty \leq X < 0 \} \\
S_{1} = \{ -\infty \leq X < 1 \} \\
S_{2} = \{ -\infty \leq X < 2 \} \\
\vdots \\
S_{x} = \{ -\infty \leq X < x \} \\$

================================================== \\
In interval $\{ a \leq x < b \}$ \\ 
simple intervale probability $P(a, b) = P(-\infty, b) - P(-\infty, a)$ \\

================================================== \\
Probability density function: \\ 
derivative of Cumulative distribution function \\ 

================================================== \\
Probability distribution function: \\ 
Probability mass function \\ 
Cumulative distribution function \\ 
Probability density function \\ 

================================================== \\

\end{document}
/home/young/Desktop/git_ssh_mltheory/mltheory/dohkim/002_Math_for_data_science/005_Probability_theory/003_Probability_distribution_function.html