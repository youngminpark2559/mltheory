\documentclass{article}
\usepackage{amsmath}
\usepackage{kotex}
\usepackage{amssymb}
\usepackage{bigints}
\begin{document}

Set: distinguishable collection. \\ 
Element: distinguishable object in the set \\ 

Element x is in set A: \\ 
$x\in A$ \\ 

================================================== \\
Cardinality is the number of element which the set has \\ 

$A=\{1,2,3\}$ \\ 
$|A| = \text{card}(A) = 3$ \\ 

================================================== \\
Union of two sets A and B \\ 
$A\cup B$ \\ 

Intersection of two sets A and B \\ 
$A\cap B$ \\ 

================================================== \\
If set A is sub set of set B: \\ 
$A \subset B$ \\ 

$A \subset A$ is true \\

If cardinality of A is smaller than B, \\ 
A is proper subset of B \\ 

================================================== \\
Difference set \\ 
$A-B$ \\ 

================================================== \\
Complement set against entire set $\Omega$ \\ 
$A^C$ \\ 

$A^C=\Omega-A$ \\ 

================================================== \\
Null set $\phi$ has none element \\ 

$\phi \subset$ all sets \\ 

Following is true \\ 
$A\cap \phi=\phi$ \\ 
$A\cup \phi=A$ \\ 
$A\cap A^C=\phi$ \\ 

================================================== \\
$A=\{1,2\}$ has 4 number of sub sets \\ 

$A_1=\phi$ \\
$A_2=\{1\}$ \\
$A_3=\{2\}$ \\
$A_4=\{1,2\}$ \\

================================================== \\
If set A has N number of elements, \\ 
A has $2^N$ number of sub sets \\ 

================================================== \\
Following is true as distribution law \\ 
$A\cup(B\cap C)=(A\cup B)\cap(A\cup C)$ \\ 
$A\cap(B\cup C)=(A\cap B)\cup(A\cap C)$ \\ 

================================================== \\

\end{document}
